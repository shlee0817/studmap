\chapter{Collector}

Der Collector ist eine Android-Applikation, die dazu dient, Daten zu vorher vom Administrator definierten Knoten/Punkten zu sammeln und 
diese entsprechend in der Datenbank zu hinterlegen. Damit ist die Applikation ein Werkzeug der Administratoren und nicht der Endbenutzer. \\
Mit der Entwicklung des Collectors haben wir eine intensive Recherche betrieben, wie wir eine Indoor-Navigation erm�glichen k�nnen.
Bevor wir das Endergebnis der Collector-Applikation n�her erl�utern, bieten wir nun einen �berblick �ber unsere Recherche-Ergebnisse.

\chapter{Recherche}
F�r die Navigation innerhalb von Geb�uden konnten wir nicht auf die g�ngigen 
Standards zur�ckgreifen, sondern mussten uns andere Wege �berlegen, wie wir 
die Nutzer innerhalb der Geb�ude lokalisieren.\\
Au�erhalb von Geb�uden ist die Lokalisierung mittels GPS sehr verbreitet und 
auch einfach und akkurat. Um eine �hnliche Lokalisierung unserer Nutzer zu 
erm�glichen haben wir uns die folgenden M�glichkeiten �berlegt.

\section{QR-Codes}
QR-Codes sind weit verbreitet, einfach zu erstellen und mit vielen Ger�ten 
einzulesen. Dadurch bieten QR-Codes die M�glichkeit Informationen einfach an 
Orten anzubringen und von Maschinen einzulesen.\\
Wir haben uns zwei Konzepte �berlegt, QR-Codes in unserem Projekt einzubauen. 
Dazu haben wir einmal die M�glichkeit betrachtet die Auswertung des QR-Codes 
am Server zu realisieren und mit der M�glichkeit verglichen die Auswertung 
direkt am Client des Nutzers zu implementieren.

\subsection{Serverseitig}
F�r die serverseitige Umsetzung hat gesprochen, dass das Smartphone keine 
Rechenleistung ben�tigt um die QR-Codes zu dekodieren. Des Weiteren wird durch 
eine serverseitige Implementierung vermieden, dass der Nutzer weitere Apps auf 
seinem Smartphone installieren muss.\\
F�r die Implementierung in den Webservice haben wir uns f�r die offene 
Bibliothek \href{http://zxingnet.codeplex.com/}{ZXing.NET} benutzt. Allerdings 
ist uns bei der Implementierung und Testen der Bibliothek direkt aufgefallen, 
dass die Bilder zuvor am Smartphone verkleinert werden m�ssen um Bandbreite zu 
sparen und die Laufzeit der Bibliothek zu verringern. Dadurch wird, obwohl es 
ein Ziel der serverseitigen Umsetzung war, Rechenleistung ben�tigt. Dar�ber 
hinaus mussten wir feststellen, dass die Bibliothek keine zuverl�ssige 
Dekodierung der QR-Codes bietet.

\subsection{Clientseitig}
Im Gegensatz zur serverseitigen Umsetzung wird bei dieser Implementierung die 
gesamte Dekodierung am Client vorgenommen. Dadurch wird die Netzlast 
verringert, der Aufwand am Client aber erh�ht.\\
Hier bot sich zum einen an eine Bibliothek in den Client aufzunehmen, oder 
eine externe Anwendung zum Dekodieren der QR-Codes zu benutzen. Wir haben uns 
schlussendlich dazu entschieden \todo{Entscheidung beim QR-Code Reader 
aufschreiben und warum}
\section{OCR der Raumschilder}
Anstatt QR- oder NFC-Tags an den R�umen der FH anzubringen, besteht die M�glichkeit
die bereits angebrachten T�rschilder zu verwenden. Hierzu ist eine Erkennung der
Raumnummer auf dem T�rschild notwendig. Die OCR (Optical Character Recognition)
kann �ber eine Bibliothek sowohl auf Client-, als auch auf Serverseite erfolgen.
Erfolgt die Erkennung auf dem Client, dann k�nnten ebenfalls bestehende Apps zur 
Texterkennung verwendet werden.

\subsection{Bibliothek tesseract}
Tesseract\footnote{\url{https://code.google.com/p/tesseract-ocr/}}
ist eine native Bibliothek f�r die Erkennung von Text in Bilddateien.
Es gibt sowohl f�r .NET als auch f�r Java (Android) entsprechende Wrapper,
die von uns verwendet werden k�nnen. Bei den Tests zu der OCR-Bibliothek haben 
sich allerdings einige Schw�chen gezeigt. Tesseract ist f�r die Texterkennung 
von gescannten Dokumenten gedacht und arbeitet deshalb nur stabil, wenn sich
auf dem Bild ausschlie�lich Text befindet. Dies wird an folgenden Beispielbildern
deutlich.

Auf dem Bild mit Raumschild und Wand wird nur unzuverl�ssig Text erkannt
(siehe \ref{fig:RaumschildGross}). Die Ergebnisse variieren von "Kein Text erkannt" bis hin zu "Buchstabensalat mit Raumnummer". Schneidet man den relevanten Teil per Hand aus (siehe \ref{fig:RaumschildKlein}), wird der Text einwandfrei erkannt.

\begin{figure}
\centering
\includegraphics[width=\linewidth]{Bilder/Raumschild_gross}
\caption{Gesamtes Bild Raumschild}
\label{fig:RaumschildGross}
\end{figure}

\begin{figure}
\centering
\includegraphics[width=\linewidth]{Bilder/Raumschild_klein}
\caption{Ausschnitt Raumschild}
\label{fig:RaumschildKlein}
\end{figure}

Damit die Raumschilder zuverl�ssig erkannt werden, ist es notwendig,
dass aufgenommene Bilder zun�chst auf den Bereich mit der Raumnummer
zugeschnitten werden. Dies erfordert einen hohen Entwicklungsaufwand.


\subsection{App Google Goggles}
Google
Goggles\footnote{\url{https://play.google.com/store/apps/details?id=com.google.android.apps.unveil}}
ist eine Android-App, die zur Erkennung von Text, Symbolen
und QR-Tags verwendet werden kann. Diese App lieferte in Tests auch bei
suboptimalen Bildern gute Ergebnisse. Au�erdem ist die App gut in das
Android-Umfeld eingebettet und l�sst sich leicht bedienen.

Allerdings gibt es f�r die Verwendung von Google Googles noch keine �ffentliche
API\footnote{\url{http://stackoverflow.com/questions/2080731/google-goggles-api}}.
Diese ist zwar von Google geplant, aber nie umgesetzt worden. Auch wenn die
App eine komfortable M�glichkeit zur OCR-Erkennung bietet, kann diese ohne
API nicht in unserem Projekt verwendet werden.



\subsection{WLAN Fingerprinting}
Bei der Positionierung des Nutzers mittels WLAN haben wir �ber eine f�r den 
Nutzer passive Positionierung recherchiert. Alle anderen 
Positionierungsmethoden ben�tigten eine Eingabe des Nutzers. Wir haben hier 
die Eingabe der Position auf einer Karte und das Einlesen von QR-Codes oder 
NFC-Tags behandelt. Die Positionierung erm�glicht es allerdings im Hintergrund 
zu laufen und ohne Eingabe des Nutzers die Position zu bestimmen.\\
WLAN ist zur Positionierung innerhalb von Geb�uden geeignet, da es zum einen 
eine weit verbreitete Infrastruktur ist, auf vielen mobilen Plattformen 
verf�gbar ist, W�nde durchdringt und Standard WLAN Access Points bereits eine 
Lokalisierung auf Raum-Genauigkeit erm�glicht.\\
Aus all diesen Gr�nden haben wir uns mit der Positionierung mittels WLAN 
besch�ftigt. 

\subsubsection{Sammeln von WLAN Fingerprints}
Um sp�ter Vergleiche im Client anstellen zu k�nnen mussten wir zuerst Daten 
des Netzwerkes sammeln. Ein Access Point wird dabei eindeutig durch eine 
\textbf{BSSID} gegenzeichnet und der Client gibt Auskunft �ber die empfangene 
Signalst�rke (\textbf{RSS \footnote{RSS: received signal strength wird in dBm 
gemessen.}}), welche beobachtet und aufgezeichnet werden kann.\\
Ziel dieser Phase war es an m�glichsten vielen Punkten in der Hochschule die 
\textit{RSS} zu messen und diese zu einem Punkt auf der Karte der Hochschule 
zu speichern.\\
\missingfigure{}

\subsubsection{Kalibrierung}
Da das Sammeln der WLAN Fingerprints mit einem Smartphone realisiert wird und 
wir davon ausgehen mussten, dass nicht jeder Nutzer das gleiche Smartphone 
besitzt, mussten wir uns eine M�glichkeit der Kalibrierung �berlegen. Dazu 
haben wir �berlegt, dass der Nutzer zuerst in einer Kalibrierungsphase selbst 
einen Fingerprint erstellt von einem von uns festgelegten Ort und diesen mit 
dem von uns gemessenen Fingerprint vergleicht. Dadurch bekommen wir einen 
Faktor um den das Smartphone des Nutzer von unserem Ger�t abweicht. Da wir 
vermuten, dass die WLAN Antennen der Smartphones auch in verschiedenen 
Bereichen, hohe, mittlere und niedrige Signalst�rke, sich stark unterscheiden 
berechnen wir diesen Faktor f�r die gerade genannten Bereiche.\\
Dieser Teilbereich ist in der \textit{StudMap-App} umgesetzt.

\subsubsection{Positionierung mittels WLAN Fingerprints}
Um die Position eines Nutzers ermitteln zu k�nnen muss dieser, wie der 
\textit{Collector} einen Fingerprint des WLANs an seiner aktuellen Position 
erstellen. Diesen Fingerprint und seine Faktoren, welche w�hrend der 
Kalibrierung ermittelt wurden, schickt der Client zum Server, welcher durch 
Vergleiche den Standpunkt ermittelt und zur�ckgibt.\todo{Wie werden die FP 
verglichen.}\\
Dieses Feature ist auch in der \textit{StudMap-App} umgesetzt.

\chapter{Positionsermittlung}
In diesem Kapitel wird beschrieben, wie die Position eines Anwenders
auf der Karte bestimmt wird. Dazu werden die im Kapitel
\nameref{cha:Recherche} beschriebenen Verfahren verwendet.

\section{QR-Tags}
\label{QR-Tags}
An den R�umen werden QR-Tags angebracht, die eine Zuordnung 
zu einem Knoten auf der erm�glicht. Hier ist zu beachten,
dass die Informationen auf dem QR-Tag auch f�r andere Anwendungen
n�tzlich sein sollen. Deshalb kann hier nicht nur eine Knoten-ID
hinterlegt werden.

Folgendes JSON-Format ist ein m�glicher Kandidat:
\begin{lstlisting}
{
  "General": {
    "Label": "A2.1.10",
    "Name": "Aquarium"
  },
  "StudMap": {
    "NodeId": "12",
    "Url": "https://code.google.com/p/studmap/"
  }
}
\end{lstlisting}



\section{Allgemeine Struktur}
\label{cha:Collector}
Die Collector-Applikation wurde als Android-Applikation umgesetzt. Es wird Android 3.0 auf einem Smartphone vorausgesetzt, um alle Funktionen nutzen k�nnen.
Anhand der Recherche-Ergebnisse haben wir uns entschieden einen Positionserkennung mittels QR-Codes, NFC-Tags und Wlan-Fingerprinting zu implementieren. 
Dabei werden QR-Codes mittels eines Tools auf dem Server generiert. Es bedarf hierbei keiner weiteren Behandlung durch die Collector-Applikation. 
Dem gegen�ber stehen das Zuordnen von NFC-Tags zu Knoten und das Sammeln von Wlan-Fingerprints durch die Collector-Applikation.
Zur Erf�llung dieser beiden Aufgaben bietet die Applikation eine schlichte Benutzeroberfl�che.

\section{Benutzeroberfl�che}
\todo{Benutzeroberfl�che und Funktionsweise sind mir fremd. Thomas und Fabian, bitte erg�nzen.}

\section{Core Bibliothek}
\label{cha:Core-Bib}
Wir haben schon fr�h festgestellt, dass es eine Vielzahl an Strukturen und Funktionalit�ten geben wird, 
die sowohl in der Collector-Applikation als auch in unserer Navigator-Applikation f�r den Endbenutzer Anwendung finden.
Dem entsprechend haben wir diese in eine eigene Android-Bibliothek ausgelagert, die wir wiederum in unsere Android-Applikationen einbinden konnten.
Zu den Kernfunktionalit�ten und -Strukturen geh�ren unter Anderem folgende:
\begin{itemize}
\item Grundlegende Definitionen einer Map, eines Floors oder eines Knoten
\item Konstanten f�r die Kommunikation mit dem Webservice
\item Ein Errorhandler f�r alle grundlegenden Fehler
\item Snippets zur einfachen Kommunikation mit dem Benutzer mittels Dialogen o.�.
\item Abbildung des Webservices zur vereinfachten Nutzung
\item Javascript-Schnittstellendefinitionen f�r die Interaktion auf der Karte
\item Asynchrone Tasks f�r Webservicekommunikation inkl. entsprechender Listener
\end{itemize}
\todo{Javascript - Daniel}
%javascript Daniel
