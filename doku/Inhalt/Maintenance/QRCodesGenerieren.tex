\section{QR-Codes generieren}
Eine M�glichkeit der Positionierung innerhalb des StudMap-Projektes ist das 
Einlesen von QR-Codes. In diesen QR-Codes stehen die im Kapitel 
\nameref{QR-Tags} beschriebenen Daten.\\
In der Anwendung werden alle \nameref{object:NodeInformation} ausgelesen und 
in einer Tabelle angezeigt. Anschlie�end kann in dieser Tabelle noch nach 
einem \nameref{object:Floor}, oder dem Namen des Raums gefiltert werden.\\
Abschlie�end kann f�r alle ausgew�hlten Knoten ein QR-Code als PNG generiert 
werden. Zur Generierung der QR-Codes nutzen wir die Bibliothek 
\href{http://qrcodenet.codeplex.com/}{QrCode.Net}.