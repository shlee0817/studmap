\chapter{Dom�nenmodell}
Durch das Dom�nenmodell legen wir Begriffe fest, mit denen
die Kommunikation im Projektteam vereinheitlicht und damit einfacher wird.

\section{Anwendungsstruktur}

\subsection*{Webservice (StudMap.Service)}
Stellt Funktionen zur Ablage und Abfrage von Navigationsinformationen und Benutzerdaten �ffentlich bereit.

\subsection*{Admin-Oberfl�che (StudMap.Admin)}
Weboberfl�che zum Anlegen, Bearbeiten von Navigationsinformationen und Benutzerdaten. Die Weboberfl�che kann nur von der Benutzerrolle Administrator bedient werden.

\subsection*{Navigations-Client (StudMap.Navigator)}
Eine Anwendung zur Anzeige von Karten und Navigation zwischen Wegpunkten. Diese wird durch den Anwender bedient.

\subsection*{Collector-Client (StudMap.Collector)}
Eine Anwendung zur Eingabe von Navigationsinformationen. Diese wird von Administratoren verwendet.
\section{Benutzerrollen}

\subsection*{Anwender (User)}
Der Anwender verwendet den Navigations-Client, um die k�rzeste Route zu einem gew�nschten Ziel zu erhalten.

\subsection*{Administrator}
Verwendet Admin-Oberfl�che und den Collector-Client. Dazu muss dieser registriert sein.

\section{Grundbegriffe}

\subsection*{Karte (Map)}
Beschreibt das gesamte Geb�ude mit allen Stockwerken.

\subsection*{Stockwerk (Floor)}
2-dimensionale Ansicht mit allen Layern der Ebene.

\subsection*{Schicht (Layer)}
Es gibt mehrere Schichten, die jeweils Detailinformationen zu
einem Stockwerk enthalten.

\begin{itemize}
\item Bild-Layer: Enth�lt grafische Darstellung des Stockwerks.
\item Graph-Layer: Enth�lt Kanten und Knoten f�r Routen.
\item POI-Layer: Zusatzinformationen zu speziellen Orten.
\item Routen-Layer: Darstellung grafischer Elemente zur Navigation.
\end{itemize}

\subsection*{Route}
Hat einen Start- und einen Endknoten. Verbindet diese beiden Knoten �ber Zwischenknoten und Kanten.

\subsection*{Graph}
Gesamtheit aller Knoten und Kanten der Karte (Stockwerk-�bergreifend).

\subsection*{Knoten (Node)}
Besteht aus eindeutigem Identifier, X- und Y-Koordinate und Stockwerk. Zu dem Knoten k�nnen zus�tzliche Informationen hinterlegt werden: Name, Raumnummer, NFC-Tag, QR-Tag und Verweis auf PoI.

\subsection*{Kante (Edge)}
Verbindung zweier Knoten. Bedeutet, dass man von einem Punkt zum anderen laufen kann.

\subsection*{Point of Interest (PoI)}
Ort besonderen Interesses (z.B. Bibliothek, Mensa, ...)
