\chapter{Einleitung}
Im Fach Fortgeschrittene Internetanwendungen haben wir uns im Rahmen einer studentischen Projektarbeit mit dem Thema Navigation und Lokalisierung innerhalb von Gebäuden beschäftigt. Dabei beschränkte sich das Ziel unseres Projekts auf die Navigation im Gebäude der Westfälischen Hochschule. Dazu stellten wir uns zu Projektbeginn eine Karte des Gebäudes vor, auf der alle möglichen Navigationsziele eingezeichnet sind. Bei der Auswahl eines Navigationsziels sollte, nach unseren Vorstellungen, eine entsprechende Wegbeschreibung eingeblendet werden, die uns von unserem aktuellen Standpunkt zum gewünschten Ziel führt.

Um dieses Ziel zu erreichen, mussten wir uns mit verschiedenen Problemstellungen auseinandersetzen. Zunächst einmal war es nötig das Gebäude vollständig in einem (unserem) System zu erfassen und dieses auf der Karte unserer Vorstellung darzustellen. Zum anderen mussten wir die aktuelle Position (innerhalb des Gebäudes) ermitteln, um diese ebenfalls auf der Karte abbilden zu können. \\


Als Plattform für unser Projekt nutzen wir hauptsächlich Google Code:\\
\href{https://code.google.com/p/studmap/}{https://code.google.com/p/studmap/}

Dort verwenden wir ein SVN Repository zur Quellcode Ablage und den Issue Tracker zur Verwaltung von Benutzeranforderungen und Fehlern. Wir haben uns in unserem Projekt für eine agile Projektorganisation nach dem Vorbild von Scrum entschieden und den Issue Tracker entsprechend konfiguriert, dass uns die Issue Typen User Story, Task und Bug zur Verfügung standen. Kurz nach Beginn des Projektes haben wir die relevanten Benutzeranforderungen in Form von User Stories angelegt.



