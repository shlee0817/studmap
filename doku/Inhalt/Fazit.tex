\chapter{Fazit}

%\section{Lokalisierung innerhalb von Gebäuden}
Im Verlauf unseres Projektes ist deutlich geworden, dass die Navigation und die Lokalisierung innerhalb von Gebäuden ein schwieriges Thema ist. Daher wählten wir für das Problem der Lokalisierung gleich mehrere Ansätze. Jedoch mussten wir erkennen, dass sowohl die Positionierung mittels Texterkennung der Raumschilder, als auch die Positionsermittlung über WLAN Finger Prints für unsere Anforderungen nicht bzw. nicht gut genug funktioniert haben.
Aus diesem Grund beschäftigten wir uns mit alternativen Möglichkeiten und entschieden uns dafür die Positionsdaten auf NFC Tags und QR Codes zu speichern und diese im Gebäude zu platzieren. So ist die zuverlässige Positionsbestimmung immer durch Scannen eines NFC Tags oder eines QR Codes möglich.

%\section{Verwendete Technologien}
Bei der Umsetzung dieser Ideen haben wir uns für ein Backend basierend auf Microsoft Technologien entschieden. Für die Entwicklung der Datenbankstruktur verwendeten wir das Entity Framework, mit dem die Datenbank automatisch aus unserem Datenmodell generiert werden konnte. Zusätzlich konnten komplexe Datenbankabfragen einfach umgesetzt und ausgewertet werden. Dadurch haben wir gerade zu Beginn des Projektes eine Menge Arbeit und Zeit gespart. Des Weiteren haben wir uns bei der administrativen Oberfläche für eine ASP.NET Webapplikation entschieden, in der wesentliche Bestandteile der Benutzeroberfläche und eine Benutzerverwaltung bereits integriert waren. Auch dadurch haben wir uns viel Arbeit erspart und konnten uns auf die Wesentlichen Probleme konzentrieren.

%\subsection{Zusammenarbeit mit Microsoft}
Begeistert von diesen vielen Möglichkeiten entschieden wir uns unsere Anwendung in der Windows Azure Cloud zu hosten und meldeten daher einen entsprechenden Studenten Account bei Microsoft an. Leider erhielten wir über Wochen kein eindeutiges Feedback von Microsoft weshalb wir uns letztendlich für einen eigenen Windows Server innerhalb der Hochschule entschieden haben.

%Im Frontend haben wir uns für eine Android Anwendung entschieden, bei der wir die wesentlichen Bestandteile in Form von Webseiten abgebildet haben. Das führte %zu einer hohen Wiederverwendbarkeit, da die Anzeige der Karte in allen Anwendung benutzt wurde.

%\section{Projektmanagement}
Abschließend blicken wir auf ein komplexes Projekt mit vielen Herausforderungen zurück. Durch die unterschiedlichen Technologien haben wir alle etwas Neues kennengelernt und weitere wichtige Erfahrung sammeln können. Innerhalb des Projekts gab es allerdings nicht nur technische Herausforderungen, auch die Projektorganisation selbst, sowie die Zusammenarbeit im Team ist in jedem Projekt eine Herausforderung. Gemeinsam haben wir es, trotz der vielen Schwierigkeiten, geschafft unser Projektziel zu erreichen. So sind insgesamt drei Anwendungen zur Navigation innerhalb unserer Hochschule entstanden, die mit entsprechenden administrativen Aufwand auch wirklich eingesetzt werden könnten.